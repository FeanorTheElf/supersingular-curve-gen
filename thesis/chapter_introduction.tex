
In recent years, much progress has been made in the construction of quantum computers that might be able to break currently used cryptographic schemes.
Hence, the search for alternative schemes is currently in the center of attention.
One approach is based on ideas of Couveignes, Rostovtsev and Stolbunov \cite{old_isogeny_crypto1, old_isogeny_crypto2, old_isogeny_crypto3}, but only became really practical with the introduction of the SIDH protocol \cite{sidh}.

The main difference to the previous protocols is that instead of ordinary Elliptic Curves, SIDH relies on supersingular curves.
These have the advantage that the structure of their rational points is much more predictable, hence allowing efficient computations.
The recent success in the cryptoanalysis of SIDH \cite{sidh_broken} is likely to end this, however our research started before that breakthrough and also studied some questions of general interest.
In particular, supersingular curves are also interesting objects on their own, and are important in the field of arithmetic geometry.

For cryptography, it is thus a natural question how to computationally generate random supersingular curves for use in various protocols.
The current approach, based on random walks in isogeny graphs (see Section~\ref{prop:naive_classical_approaches}), is very efficient, but has a drawback.
Namely, the computation reveals an isogeny path to the starting curve, which can be used for attacks in some scenarios.
It is currently an open problem if we can generate Elliptic Curves efficiently in a way that does not reveal such an isogeny, nor other information that might be used for attacks in certain scenarios.
Up to now, in these scenarios, a trusted third party is thus required, that can uses a classical method to generate a curve, and then forgets about the additional information produced in the process.

Some ideas to solve this have been proposed in \cite{base_paper}, most of them trying to exploit special structure to find roots of very large polynomials.
However, for each approach so far there are some serious obstacles before it might be practical.
In this work, we focus on the second of these, which is based on a correlation between a curve being supersingular, and having two fixed-degree isogenies to its Frobenius conjugate.
The idea is now to find a random curve with two such isogenies, which then is supersingular with high probability.
Finding a curve with these isogenies is equivalent to finding a root of a certain polynomial system involving modular polynomials, which might be possible to do computationally.

Our work studies this correlation, and finds a special case in which we can quantify it.
Namely, we prove the following statement.
\begin{prop}[First Result]
    Let $l$ be a small prime, $f$ be an odd integer and $e$ an even integer such that $l^f$ is polynomial in $\log(p)$ and $l^e = \Theta(p)$.
    Then a random Elliptic Curve over $\bar{\F}_p$ with a cyclic $l^f$-isogeny and any $l^e$-isogeny to its Frobenius conjugate is supersingular with exponentially high probability (in $\log(p)$).
\end{prop}
Furthermore, we also propose a modified idea, and explain why it might be easier to implement efficiently.
\begin{prop}[Second Result]
    Let $l_1, ..., l_r$ be a small primes with $\prod l_i \geq \sqrt{p}$.
    Then a random Elliptic Curve over $\F_{p^2}$ such that there are three $l_i$-isogeneous curves over $\F_{p^2}$ for each $i$ is supersingular with exponentially high probability.
\end{prop}
Finally, we also want to present in detail some results that transfers classical results in number theory to the statements required in cryptography.
In particular, most literature in number theory deals with curves over $\C$, while in crypto we work in the finite field setting.
Moreover, the very influential papers \cite{deuring_endomorphism_rings} and \cite{class_group_action_waterhouse} are not very accessible, as the first one is written in German, and the second works in the much more general setting of abelian varieties.
Hence, we perceive in the literature a tiny gap between algebraic geometric foundations and the statements used in crypto, which we want to bridge in this work.