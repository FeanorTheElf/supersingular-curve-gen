The continuing progress in the construction of quantum computers has pushed the development of post-quantum cryptographic schemes into the center of attention.
One approach is isogeny-based cryptography, which relies on the theory of elliptic curves and their isogenies.
Founded on ideas of Couveignes, Rostovtsev and Stolbunov \cite{old_isogeny_crypto1, old_isogeny_crypto2, old_isogeny_crypto3}, a variety of different schemes have been proposed since then.
This includes the well-known, although recently broken \cite{sidh_broken}, key exchange protocol SIDH \cite{sidh}.

Although the pioneering works have been based on ordinary curves, a majority of later schemes uses supersingular curves.
Hence, it is a natural question how to computationally generate supersingular curves in implementations.
The good new is that methods based on complex multiplication \cite{constructing_supersingular_curves} together with random walks allow us to find uniformly random supersingular curves even over prime fields of exponentially large characteristic.
However, this method has a drawback.
Namely, whoever generates a curve using this algorithm can easily find its endomorphism ring, which can be used as a trapdoor for various cryptographically relevant problems \cite{endomorphism_ring_isogeny_path_equivalent}.

In many cases like encryption, this is not a problem at all, since the person generating the curve has access to the secret anyway, for example because they are a legitimate party in the communication.
However, even for SIDH (before it was broken), there were some subtle security problem related to the torsion-point attacks, that could be prevented by other, trapdoor-free ways of finding starting curves.
Furthermore, many applications, for example in blockchain environments or for more sophisticated primitives \cite{verifiable_delay_function, torsion_point_problem2}, are insecure if any party has knowledge of a trapdoor for the starting curve.
Hence, these scenarios currently require a trusted third party, that generates a curve using one of the known approaches, and then forgets about the additional information produced in the process.
Therefore, there is natural interest in methods to eliminate this trusted third party, by finding algorithms that generate supersingular curves, for whom the endomorphism ring problem is as hard as for random curves - even when the randomness used for the generation is known.
This is currently an open problem.

\paragraph{Katherine Stange's idea} Some approaches have been proposed in \cite{base_paper} and also in \cite{concurrent_paper}, most of them trying to exploit special structure to find roots of very large polynomials.
However, for each approach so far there are some serious obstacles that must be overcome before it might be practical.
In this work, we focus on the second idea from \cite{base_paper}, which is proposed by Katherine Stange.
Basically, it relies on the observation that Elliptic Curves with fixed-degree isogenies to their Galois conjugate are supersingular with higher probability.
Further, they propose an approach based on modular polynomials and resultants that can find a random curve with two isogenies of different, fixed degree to their Galois conjugate.
However, as mentioned in \cite{base_paper}, there are two main problems with this approach.

First, it is not clear how strong the correlation between having fixed-degree isogenies to the conjugate and the supersingularity is.
The paper contains an estimate under the assumption that the existence of isogenies is in a certain sense independent, but this estimate does not completely match their experimental data.
Furthermore, in the case of taking two isogenies of different degree, the correlation seems to be too weak for the idea to work properly.
According to their heuristic, it can be fixed by using three different isogenies, but this is also not proven, and computationally more expensive than the two-isogeny variant.

The second problem is that in order to avoid vulnerabilities, the algorithm has to work with modular polynomials of exponential degree.
Currently, no way to exploit special structure is known that would allow us to do this efficiently.

\paragraph{Our contribution} In our research, we tried to address both problems.
Namely, we were able to find a special case of the two-isogeny variant, in which the fraction of supersingular Elliptic Curves is provably big enough.
More concretely, we present the following result.
\begin{prop}[First Result]
    \label{prop:main_result1}
    Let $l$ be a small prime, $f$ be an odd integer and $e$ an even integer such that $l^e = \Theta(p)$.
    Then a random Elliptic Curve over $\F_{p^2}$ with a cyclic $l^f$-isogeny and any $l^e$-isogeny to its Frobenius conjugate is supersingular with probability exponentially close to 1.
\end{prop}
Taking the degrees of the isogenies to be prime powers might additionally have computational advantages, as it allows us to decompose the isogeny into a sequence of smaller ones.

The second problem seems to be more difficult, and we did not find an algorithm that is efficient enough.
However, we also propose a variant of the original idea, and argue that the structure of the corresponding polynomials looks like it might make computations simpler.
This new method is based on the following statement, which is our second main result.
\begin{prop}[Second Result]
    \label{prop:main_result2}
    Let $l_1, ..., l_r$ be a small primes with $\prod l_i \geq 2p$.
    Then a random Elliptic Curve over $\bar{\F}_p$ such that there are three $l_i$-isogenous curves over $\F_{p^2}$ for each $i$ is supersingular with probability exponentially close to 1.
\end{prop}
Finally, we also present some classical results from the theory underlying isogeny graphs, in the hope of making them more accessible to cryptographers.
Most of the standard mathematical literature on the subject (e.g. \cite{cox_primes_of_form}) usually focuses on the case of Elliptic Curves over $\C$, and the finite field setting used in cryptography introduces some additional subtleties.
The finite field setting and its connection to the classical, complex setting are rarely treated, and then in works like \cite{deuring_endomorphism_rings} or \cite{class_group_action_waterhouse}, which are quite challenging.
For example, the work of Deuring \cite{deuring_endomorphism_rings} is quite old and written in German, while the work of Waterhouse \cite{class_group_action_waterhouse} treats the much more general theory of abelian varieties, and uses a great deal more algebraic geometry than necessary for Elliptic Curves. 
To summarize, (relatively) elementary proofs for some classical results seem to be missing in the crypto literature, and we also want to bridge this gap in this work.