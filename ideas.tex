\documentclass{scrartcl}

\usepackage{graphicx}
\usepackage[utf8]{inputenc}
\usepackage[T1]{fontenc}
\usepackage{lmodern}
\usepackage[english]{babel}
\usepackage{amsmath}
\usepackage{amsthm}
\usepackage{mathtools}
\usepackage{amssymb}
\usepackage{listings}
\usepackage{xparse}
\usepackage{geometry}
\usepackage{enumerate}
\usepackage{tikz}
\usepackage{stmaryrd}
\usepackage[style=english]{csquotes}
\usepackage[language=english, backend=biber, style=alphabetic, sorting=nyt]{biblatex}

\usetikzlibrary{babel, positioning, shapes.geometric, arrows, arrows.meta}
\addbibresource{bibliography.bib}

\title{Ideas}
\author{Simon Pohmann}

\newcommand{\N}{\mathbb{N}}
\newcommand{\Z}{\mathbb{Z}}
\newcommand{\F}{\mathbb{F}}
\newcommand{\C}{\mathbb{C}}
\newcommand{\I}{\mathbb{I}}
\newcommand{\V}{\mathbb{V}}
\newcommand{\End}{\mathrm{End}}
\newcommand{\proj}{\mathrm{proj}}
\newcommand{\Quot}{\mathrm{Quot}}
\newcommand{\Half}{\mathcal{H}}
\newcommand{\Lattice}{\mathcal{L}}
\newcommand{\divides}{\ \mid \ }
\newcommand{\notdivides}{\ \nmid \ }
\newcommand{\Cl}{\mathrm{Cl}}
\newcommand{\K}{\mathcal{K}}
\renewcommand{\a}{\mathfrak{a}}
\renewcommand{\b}{\mathfrak{b}}
\renewcommand{\O}{\mathcal{O}}
\renewcommand{\N}{\mathfrak{N}}

\newcommand\restr[2]{{
    \left.\kern-\nulldelimiterspace
    #1
    \vphantom{\big|}
    \right|_{#2}
}}

\newtheorem{prop}{Proposition}[section]
\newtheorem{theorem}[prop]{Theorem}
\newtheorem{lemma}[prop]{Lemma}
\newtheorem{corollary}[prop]{Corollary}

\theoremstyle{definition}
\newtheorem{problem}[prop]{Problem}
\newtheorem{alg}[prop]{Algorithm}
\newtheorem{definition}[prop]{Definition}
\newtheorem{example}[prop]{Example}
\newtheorem{remark}[prop]{Remark}

\begin{document}

\maketitle
\tableofcontents

\section{My first idea}
As usual, let $p$ be a (big) prime and consider $q := p^2$.
Consider a (small) prime $l$.
Then every supersingular Elliptic Curve $E/\F_q$ satisfies $\Phi_n(j(E), j(E)^p) = 0$ with $n = l^{O(\log(p))}$, as the supersingular $l$-isogeny graph is an expander with mixing length $O(\log(p))$, hence there is a path from $E$ to $E^{(p)}$ of length $O(\log(p))$.

Now we analyze when $\Phi_n(j(E), j(E)^p) = 0$ for an ordinary Elliptic Curve $E/\F_q$.

\subsection*{Using the isogeny graph}
Since the connected component of $E$ in the $l$-isogeny graph is a vulcano, we can find a path (of length $O(\log(p))$) to an Elliptic Curve in the crater, say $E_0$.
Hence there are ascending $l$-isogenies
\begin{equation*}
    E \to ... \to E_0
\end{equation*}
Let $K := \End^0(E_0)$ and consider the maximal order $\O_K \subseteq K$, $\O_0 := \End(E_0)$ and $\O := \End(E)$.
Then have that $\O \subseteq \O_0 \subseteq \O_K$ with $[\O_0 : \O] = l^{O(\log(p))}$ and $l \notdivides [\O_K : \O_0]$.

Now we are in one of the following cases:
\begin{description}
    \item[(I) - ``bad''] $E_0$ is defined over $\F_p$, i.e. $E_0^{(p)} = E_0$; Then $\Phi_n(j(E), j(E)^p) = 0$
    \item[(II) - ``probably good''] $E_0^{(p)}$ is (nontrivially) $l$-isogeneous to $E_0$, i.e. they are two distinct vertices on the crater; Then it is likely that $\Phi_n(j(E), j(E)^p) \neq 0$, but that depends on the distance in the crater
    \item[(III) - ``good''] $E_0^{(p)}$ is not $l$-isogeneous to $E_0$; Then $\Phi_n(j(E), j(E)^p) \neq 0$ 
\end{description}

\subsection{Analyzing (III)}
Now consider only $E_0$ and denote $\O := \O_0$ and $E := E_0$.

Let $[\a] \in \Cl(\O)$ such that $[\a].E = E^{(p)}$.
We have (III) if and only if $[\a]$ contains no integral ideal of index $l^r$, for any $r \in \N$.
Assume it does, say $\b$.
Then $\b = \alpha\a$ for some $\alpha \in \a^{-1}$ with
\begin{equation*}
    \N(\alpha) = \frac {\N(\b)} {\N(\a)} = \frac {l^r} {\N(\a)}
\end{equation*}
Since we do not require $\alpha$ to be integral, we can substitute $\alpha$ by $\alpha/l$ and so find that
\begin{equation*}
    \N(\alpha) = \N(\a)^{-1} \quad \text{or} \quad \N(\alpha) = l\N(\a)^{-1}
\end{equation*}
This leads us to the interesting (slightly weaker) question: When does there exist some $\alpha \in K$ with $\N(\alpha) = N$ for some (square-free) $N$?

\subsection{Preventing (I)}
Consider an ordinary Elliptic Curve $E/\F_{p^2}$ and let $\pi$ be the $q = p^2$-th power Frobenius.
Then $\pi$ satisfies $\pi^2 - t\pi + q = 0$ where $t = q + 1 - \#E(\F_q)$ is the trace of Frobenius.
By the Hasse bound, we derive the standard bound
\begin{equation*}
    |t| \leq 2\sqrt{q} = 2p
\end{equation*}
Now consider the Endomorphism ring $\O = \End(E)$ in the number field $\K := \O \otimes \mathbb{Q}$.
We have that $\Z[\pi] \subseteq \O$ and so $[\O_\K : \O] \leq [\O_\K : \Z[\pi]]$.
Now note that the discriminant of $\Z[\pi]$ is
\begin{equation*}
    D := d(\Z[\pi]) = t^2 - 4q
\end{equation*}
and so $-4q \leq D \leq 0$\footnote{In fact, since $E$ is ordinary, we have $D \not\equiv 0 \mod p$, but that does not matter here.}.
Now we use the standard fact on modules
\begin{equation*}
    d(\O_\K) [\O_\K : \Z[\pi]]^2 = d(\Z[\pi])
\end{equation*}
and find that $[\O_\K : \Z[\pi]] \leq \sqrt{|d(\Z[\pi])|} = 2p$.
After these preliminaries, we can consider the approach itself.

\subsection{Choosing parameters}
Let $l$ be a prime $> 2p$ and $e = O(\log(p))$.
Now consider a root $j \in \F_{p^2} \setminus \F_p$ of $\Phi_{l^e}(X, X^p)$ and an Elliptic Curve $E$ with j-invariant $j$.
We claim that case I cannot occur.

Suppose $E$ is ordinary.
Then there is the standard ascending chain of $l$-isogenies
\begin{equation*}
    E \to ... \to E_0
\end{equation*}
such that $E_0$ lies on the crater, i.e. $l \notdivides [\O_\K : \End(E_0)]$ where $\K = \End^0(E) = \End^0(E_0)$.
However, since $[\O_\K : \End(E)] \leq [\O_\K : \Z[\pi]] \leq 2p$ where $\pi$ is the $q$-th power Frobenius of $E$, we see that $l \notdivides [\O_\K : \End(E)]$.
Thus $E = E_0$.

However, we chose $j \notin \F_p$ and so $E = E_0$ is not defined over $\F_p$.
Thus $E_0^{(p)} \neq E_0$, which excludes case I.

\section*{Current idea}
I still have to make (much) more experiments, but somehow, case II seems strange - possibly it does not even occur (I have not encountered it so far).
If this is really the case (or if it is really rare), then we can just choose $l > 2p$ prime and $e = O(\log(p))$ and find a root $j \in \F_{p^2} \setminus \F_p$ of $\Phi_{l^e}(X, X^p)$.

In other words, compute $F := \Phi_{l^e}(X, X^p) \mod X^q - X$ and then find a root of
\begin{equation*}
    \frac {F} {\gcd(F, X^p - X)}
\end{equation*}
Of course, $\deg(F) = \Theta(\log(p))$ is still exponential, so it will not work out easily...
\end{document}