\documentclass{scrartcl}

\usepackage{graphicx}
\usepackage[utf8]{inputenc}
\usepackage[T1]{fontenc}
\usepackage{lmodern}
\usepackage[english]{babel}
\usepackage{amsmath}
\usepackage{amsthm}
\usepackage{mathtools}
\usepackage{amssymb}
\usepackage{listings}
\usepackage{xparse}
\usepackage{geometry}
\usepackage{enumerate}
\usepackage{tikz}
\usepackage{stmaryrd}
\usepackage[style=english]{csquotes}
\usepackage[language=english, backend=biber, style=alphabetic, sorting=nyt]{biblatex}

\usetikzlibrary{babel, positioning, shapes.geometric, arrows, arrows.meta}
\addbibresource{bibliography.bib}

\title{Ideas}
\author{Simon Pohmann}

\newcommand{\N}{\mathbb{N}}
\newcommand{\Z}{\mathbb{Z}}
\newcommand{\F}{\mathbb{F}}
\newcommand{\C}{\mathbb{C}}
\newcommand{\I}{\mathbb{I}}
\newcommand{\V}{\mathbb{V}}
\newcommand{\End}{\mathrm{End}}
\newcommand{\proj}{\mathrm{proj}}
\newcommand{\Quot}{\mathrm{Quot}}
\newcommand{\Half}{\mathcal{H}}
\newcommand{\Lattice}{\mathcal{L}}
\newcommand{\divides}{\ \mid \ }
\newcommand{\notdivides}{\ \nmid \ }
\newcommand{\Cl}{\mathrm{Cl}}
\renewcommand{\a}{\mathfrak{a}}
\renewcommand{\b}{\mathfrak{b}}
\renewcommand{\O}{\mathcal{O}}
\renewcommand{\N}{\mathfrak{N}}

\newcommand\restr[2]{{
    \left.\kern-\nulldelimiterspace
    #1
    \vphantom{\big|}
    \right|_{#2}
}}

\newtheorem{prop}{Proposition}[section]
\newtheorem{theorem}[prop]{Theorem}
\newtheorem{lemma}[prop]{Lemma}
\newtheorem{corollary}[prop]{Corollary}

\theoremstyle{definition}
\newtheorem{problem}[prop]{Problem}
\newtheorem{alg}[prop]{Algorithm}
\newtheorem{definition}[prop]{Definition}
\newtheorem{example}[prop]{Example}
\newtheorem{remark}[prop]{Remark}

\begin{document}

\maketitle

\section{$(d, \epsilon)$-structures}

Let $p$ be a prime.
Consider the category $\mathrm{EC}$ defined by
\begin{align*}
    \mathrm{Ob}(\mathrm{EC}) &:= \{ \text{$E$ elliptic curve over $\F_{p^2}$} \} \\
    \mathrm{Hom}_{\mathrm{EC}}(E, E') &:= \{ \text{$\psi: E \to E'$ isogeny} \}
\end{align*}
Have a functor
\begin{align*}
    \cdot^{(p)}: \mathrm{EC} \ &\to \ \mathrm{EC} \\
    \text{$E$ defined by $y^2 = x^3 + Ax + B$} \ &\mapsto \ \text{$E'$ defined by $y^2 = x^3 + A^p x + B^p$} \\
    \left[ \sum_{i, j} a_{ij} x^i y^j : \sum_{i, j} b_{ij} x^i y^j : \sum_{i, j} c_{ij} x^i y^j \right] \ &\mapsto \ \left[ \sum_{i, j} a_{ij}^p x^i y^j : \sum_{i, j} b_{ij}^p x^i y^j : \sum_{i, j} c_{ij}^p x^i y^j \right]
\end{align*}
and a functor
\begin{align*}
    \hat{\cdot}: \mathrm{EC} \ \to \ \mathrm{EC}^{\mathrm{op}}, \quad E \ \mapsto \ E, \quad \phi \ \mapsto \ \hat{\phi}
\end{align*}
$(d, \epsilon)$-structures and their isogenies are given by the category $\mathrm{ES}_{d, \epsilon}$ defined by
\begin{align*}
    \mathrm{Ob}(\mathrm{ES}) &:= \{ (E, \psi) \ | \ E \in \mathrm{EC}, \ \psi: E \to E^{(p)}, \ \hat{\psi} = \epsilon\psi^{(p)} \} \\
    \mathrm{Hom}_{\mathrm{ES}}((E, \psi), (E', \psi')) &:= \{ \phi: E \to E' \ | \ \psi' \circ \phi = \phi^{(p)} \circ \psi \}
\end{align*}

\section{$j$-invariant and modular polynomials}

Consider the $j$-invariant
\begin{equation*}
    j: \Half \to \C
\end{equation*}
that assigns to a complex elliptic curve given by a lattice $\Lattice\{\tau, 1\}$ its j-invariant $j(\tau)$.
Then it is a fact that for $N \in \N$ the map
\begin{equation*}
    j_N: \Half \to \C, \quad \tau \mapsto j(N\tau)
\end{equation*}
is algebraic over $\C(j)$ and its minimal polynomial is $\Phi_N(X, j)$.
This $\Phi_N$ is called modular polynomial, and we have $\Phi_N \in \mathbb{Q}[X, Y]$ and furthermore $\Phi_N(X, Y) = \Phi_N(Y, X)$.

Furthermore, it holds that
\begin{equation*}
    \Phi_N(j(E), j(E')) = 0
\end{equation*}
for any $E'$ such that there is an $N$-isogeny $E \to E'$ (No idea how to prove that).

We see then that for all primes $p$, have
\begin{equation*}
    \Phi_N(j(E), j(E')) = 0
\end{equation*}
for elliptic curves $E, E'$ defined over $\bar{\F}_p$ such that there is an $N$-isogeny $E \to E'$.

This shows that if we have a $(d, \epsilon)$-structure $(E, \psi)$ then
\begin{equation*}
    \Phi_d(j(E), j(E^{(p)})) = \Phi_d(j(E), j(E)^p) = 0
\end{equation*}
as there is the $d$-isogeny $\psi: E \to E^{(p)}$.

\section{My first idea}
As usual, let $p$ be a (big) prime and consider $q := p^2$.
Consider a (small) prime $l$.
Then every supersingular Elliptic Curve $E/\F_q$ satisfies $\Phi_n(j(E), j(E)^p) = 0$ with $n = l^{O(\log(p))}$, as the supersingular $l$-isogeny graph is an expander with mixing length $O(\log(p))$, hence there is a path from $E$ to $E^{(p)}$ of length $O(\log(p))$.

Now we analyze when $\Phi_n(j(E), j(E)^p) = 0$ for an ordinary Elliptic Curve $E/\F_q$.

\subsection*{Using the isogeny graph}
Since the connected component of $E$ in the $l$-isogeny graph is a vulcano, we can find a path (of length $O(\log(p))$) to an Elliptic Curve in the crater, say $E_0$.
Hence there are ascending $l$-isogenies
\begin{equation*}
    E \to ... \to E_0
\end{equation*}
Let $K := \End^0(E_0)$ and consider the maximal order $\O_K \subseteq K$, $\O_0 := \End(E_0)$ and $\O := \End(E)$.
Then have that $\O \subseteq \O_0 \subseteq \O_K$ with $[\O_0 : \O] = l^{O(\log(p))}$ and $l \notdivides [\O_K : \O_0]$.

Now we are in one of the following cases:
\begin{description}
    \item[(I)] $E_0$ is defined over $\F_p$, i.e. $E_0^{(p)} = E_0$; Then $\Phi_n(j(E), j(E)^p) = 0$
    \item[(II)] $E_0^{(p)}$ is (nontrivially) $l$-isogeneous to $E_0$, i.e. they are two distinct vertices on the crater; Then it is likely that $\Phi_n(j(E), j(E)^p) \neq 0$, but that depends on the distance in the crater
    \item[(III)] $E_0^{(p)}$ is not $l$-isogeneous to $E_0$; Then $\Phi_n(j(E), j(E)^p) \neq 0$ 
\end{description}

\subsection*{Analyzing (III)}
Now consider only $E_0$ and denote $\O := \O_0$ and $E := E_0$.

Let $[\a] \in \Cl(\O)$ such that $[\a].E = E^{(p)}$.
We have (III) if and only if $[\a]$ contains no integral ideal of index $l^r$, for any $r \in \N$.
Assume it does, say $\b$.
Then $\b = \alpha\a$ for some $\alpha \in \a^{-1}$ with
\begin{equation*}
    \N(\alpha) = \frac {\N(\b)} {\N(\a)} = \frac {l^r} {\N(\a)}
\end{equation*}
Since we do not require $\alpha$ to be integral, we can substitute $\alpha$ by $\alpha/l$ and so find that
\begin{equation*}
    \N(\alpha) = \N(\a)^{-1} \quad \text{or} \quad \N(\alpha) = l\N(\a)^{-1}
\end{equation*}
This leads us to the interesting (slightly weaker) question: When does there exist some $\alpha \in K$ with $\N(\alpha) = N$ for some (square-free) $N$?
\end{document}